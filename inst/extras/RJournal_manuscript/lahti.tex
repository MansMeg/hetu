% !TeX root = RJwrapper.tex
\title{Validating and Extracting Information from National Identification Numbers in R: the case of Finland and Sweden}
\author{by Pyry Kantanen, Erik Bulow, Måns Magnusson, Leo Lahti} % Add affiliations?

\maketitle

\abstract{
National identification numbers (NIN) are a widely used method for uniquely identifying individuals and companies in Finland, Sweden, and many other countries. Their widespread use in both public sector registers and private sector customer data systems prompts the need for openly available methods and tools for NIN analysis and validation. The \pkg{hetu} and \pkg{sweidnumbr} R packages provide functions for extracting information from Finnish and Swedish NINs. The packages can also generate temporary but structurally valid random numbers for testing purposes. Finnish and Swedish NINs contain information about person's birth date and sex, an individual number and a control character that helps detect most common input errors and structural irregularities. Our packages contain comprehensive documentation and examples to make usage of the package easy for domain experts with non-technical backgrounds. While the introduced tools are relevant mainly in the Finnish and Swedish context, this work contributes to developing similar tools for other countries. From a wider perspective, these packages contribute to the growing toolkit of standardized methods for computational social science research, epidemiology and other register-based inquiries.}

\section{Introduction}

The \pkg{hetu} and \pkg{sweidnumbr} R packages provide open tools for handling and extracting data from identification codes for natural persons and juridical persons in the context of Finland and Sweden. Identification codes for natural persons are called \dfn{national identification numbers} (NIN) throughout this manuscript. NINs in the Nordic countries, a group which Finland and Sweden are a part of, contain embedded information that can be extracted and analysed.

Technical systems for identifying people, organizations, places and other objects are an important but often overlooked aspect of governance and management tools in modern societies \citep{dodge2005}. Universal and persistent identification numbering systems for natural persons are vital facilitating research activities that combine data from different sources, for example in the fields of epidemiology, population studies and social research \citep{gissler2004}. Outside the field of academic research, universal identifiers for natural persons enables work in a multi-disciplinary and multi-agency contexts due to greater \dfn{administrative fluency} and \dfn{bureaucratic effectiveness} \citep{alastalo2022}. This may be useful for example in the case of tackling wicked social problems that require co-operation from professionals from various fields, such as social workers, psychologists, police and health care professionals.

Prior R packages with similar scope include \CRANpkg{numbersBR} for Brazilian identity numbers for individuals, vehicles and organisations \citep{numbersbr}, and \CRANpkg{generator} for generating various types of Personally Identifiable Information (PII), such as fake e-mail addresses, names and United States Social Security Numbers \citep{generator}.

NIN generators and validators are therefore not a novel concept. In fact, in the case of Finland, handling Finnish NINs seems to be a common entry-level task to familiarize new computer science students with regular expressions, dates, string subsetting and similar concepts. The reasoning behind packaging these functions as R packages are similar to the ones listed in Wickham and Bryan's R Packages Introduction: code sharing, transparency, user manuals, semantic versioning and documenting changes between versions \citep{wickham2022}. In addition to these generally accepted virtues of open source software, there is of course intrinsic value in providing a solution to a common problem in the R language.

\section{Features of identification number systems}

Reliably keeping track of individuals, organizations, objects and flows in a given territory has long been seen as an important feature of modern governance \citep{dodge2005}. \citet[115-120]{foucault7778} observed a historical pattern where practices first implemented in disciplinary institutions, such as prisons, military units and schools have spread to influence whole societies. We can see the results of this development today in the form of different identification code systems implemented around the world. Differences in legal, political and historical frameworks in different countries have affected how these systems are implemented in practice, causing heterogeneity for example in identification system designs across Europe \citep{otjacques2007}. 

This heterogeneity as well as linguistic differences seem to contribute to variance in terminology used when referring to identification code systems. \dfn{Unique identifier} (UID) is an umbrella term that can be used to refer to unique identifiers for all sorts of things, from books (ISBN), chemicals (CAS) and legal entities (LEI) to anything imaginable \citep[see][]{dodge2005}. In this paper we are mainly interested in unique identifiers for natural persons and juridical persons. 

Names such as \dfn{personal identification code} \citep{dodge2005}, \dfn{personal identity number} \citep{alastalo2022} and \dfn{single identification number} (SIN) \citep{otjacques2007} are used in literature as a generic term. Personal identification codes can sometimes be confused with personal \dfn{personal identification numbers} (PIN) that refer to numeric or alphanumeric passcodes used for authentication, for example to withdraw cash or open a locked mobile phone. On the other hand names such as \dfn{personal identity code} (PIC) \citep{hetudvv, sund2012}, name number \citep{watson2010} and personal number \citep{scb2016} are used in official translations to refer to national implementations of NINs; in the mentioned cases, Finnish, Icelandic and Swedish NINs, respectively. Due to \dfn{function creep} \citep[see][]{brensinger2021, alastalo2022} or \dfn{control creep} \citep[see][]{dodge2005}, historically sector-specific identifiers may also be used as a \emph{de facto} NIN. This is the case with the US \dfn{social security number} (SSN) \citep{brensinger2021} and Finnish employee pension card numbers and social security codes \citep{alastalo2022}.

For clarity's sake, we will be using the generic term \dfn{national identification number} (NIN) throughout the manuscript to refer to all identification number systems and their implementations for natural persons, Finnish \dfn{personal identity codes} and Swedish \dfn{personal number} in particular. For organizations, we will use the generic term \dfn{organization identifier} when discussing Finnish \dfn{business IDs} (BID) and Swedish \dfn{organizational identity numbers} (OIN) / or Swedish organizational number (SON). 

All identification code systems should strive to be both \dfn{unique} and \dfn{self-same} over time. Self-sameness refers to a degree of immutability that allows organizations to identify and reidentify a person over time. A combination of attributes\footnote{Think of SQL's \dfn{composite key}} such as name, occupation and address would probably form a unique identifier even in relatively large crowds, but such attributes might not stay the same over time. \citep{brensinger2021}. In everyday life settings, a combination of personal attributes, such as name, occupation and address may be sufficiently unique to identify individuals even in large groups of people. Such an approach has its limitations. The problem with such composite keys is, however, that parts that make the composite key might not be permanent, making it impossible to compare different data points in the same dataset or combining data from different datasets.

According to \citet{Alterman2003} a distinction can be made between \dfn{biocentric data} and \dfn{indexical data}. The former is biometric data connected to the physical features of the individual whereas the latter has no distinguishable relation to the individual, physiologically, psychologically, or otherwise. An example of biocentric data could be a fingerprint or an iris scan and an example of indexical data could be a randomly assigned number from which nothing can be deduced.\footnote{\citet{brensinger2021} discuss the concept of \dfn{dividuals}, manufactured objects that represent the living individual: address, fingerprints, name and so on. These dividuals need to go through the process of disembedding, standardization and reembedding to be useful. Disembedding means data gathering (taking a fingerprint sample), standardization means making the disembedded transcription into a standardized digital sample that can be easily compared with other similar samples and reembedding means linking these standardized records back to their actual flesh-and-blood counterparts. Without a way to reembed a huge and well standardized archive of fingerprints back to the population it is essentially useless. This is also a reason why biometric samples such as iris scans or fingerprints can never replace a primary keys in databases.}

For a number of reasons, many identification numbers are not just random strings. The American SSN originally contained information about the person's birth year and where the number was first registered \citep[32]{brensinger2021} whereas Nordic countries' NINs often contain (or used to contain) information about the individual, usually birth date and sex \citep{watson2010, salste2021}. One reasoning for this was to make the code easier to remember \citep{alastalo2022}. This feature takes Nordic NINs closer to the definition of biocentric data than indexical data, making them more sensitive to handle. Table 1 provides a summary on the introduction of NINs in the Nordic countries as well as information which they contain.

\begin{table}[ht]
\centering
\begin{tabular}{lllllll}
\toprule
  country & NIN name & introduced & characters (n) & birth date & sex & birth place \\
  \hline
  Sweden & personnummer & 1947 & 11 & yes & yes & yes\\
  Iceland & kennitala & 1950 & 10 & yes & no & no \\
  Norway & fødselsnummer & 1964 & 11 & yes & yes & no \\
  Denmark & CPR-nummer & 1968 & 11 & yes & yes & no \\
  Finland & henkilötunnus & 1968 & 11 & yes & yes & no \\
\bottomrule
\end{tabular}
\caption{Nordic NINs: year introduced and embedded information.}
\label{tab:nordiccomparison}
\end{table}

In the Nordic countries comprehensive national identification number systems were developed and implemented from the 1940's to 1960's \citep{watson2010}. In Sweden, the personal identity number (PIN) was introduced in 1947 as a way to identify individuals for tax purposes. The personal identity number consisted both of the date of birth and an additional three-digit birth number that together uniquely identified every individual in Sweden. In 1967 a check digit was added to easier identify incorrect entries to finalize the full Swedish personal identity number (or \dfn{personnummer}) \citep{johansson2003,scb2016}. 

The Finnish personal identity code has its roots in specialized employment pension number from 1962, which was gradually expanded to cover the whole population in the form of a social security number in 1964-1968. The structure of employment pension number was designed by a Finnish mathematician Erkki Pale, who had worked in Sweden in 1948-1951 and was most likely inspired by the Swedish regional birth numbers \citep{alastalo2022}. His contribution was to design a more robust control character system to mitigate for input errors, especially in punch card systems. The design has proved to be resilient and with some minor tweaks it continues to be used, with the modern iteration being called a \dfn{personal identity code} (In Finnish: henkilötunnus, or \dfn{hetu} for short, hence the name of the package) \citep{salste2021}. Similar to Finland, other Nordic countries took inspiration from Sweden as well \citep{Krogness2011}. Table 2 illustrates the similar structures of NINs in different Nordic countries.

\begin{table}[ht]
\centering
\begin{tabular}{llll}
\toprule
  country & NIN name & NIN example & NIN structure \\
  \hline
  SE & personnummer & 610321-3499 & YYMMDDCNNNQ \\
  IS & kennitala & 121212-1239 & DDMMYYNNQC \\
  NO & fødselsnummer & 110779 41012 & DDMMYYNNNQQ \\
  DK & CPR-nummer & 300280-1178 & DDMMYY-NNNQ \\
  FI & henkilötunnus & 131052-308T & DDMMYYCNNNQ \\
\bottomrule
\end{tabular}
\caption{Examples of national identification numbers and their composition in five Nordic countries. DD: day, MM: month, YY: year, C: century marker, N: personal number numerical digit, Q: check digit or a control character.}
\label{tab:nordiccomparison2}
\end{table}

In Finland the expansion of sector specific social security numbers and employment pension numbers to universal NINs in 1969 has contributed to the widespread use of secondary data sources\footnote{Secondary data: Data that has not been collected primarily for the purpose of a specific research question} in research. One example of this development is the use of Finnish Hospital Discharge Register (FHDR), which contains data on all inpatient hospital discharges since 1967. According to \citet{sund2012} the proportion of erroneously inputed NINs or incomplete pseudo-NINs was initially high but fell rapidly as hospitals were specifically instructed to include it systematically and correctly to all discharge records starting from 1972. This has allowed to link discharge register data to other sources. In Sweden the PIN is currently used extensively in all parts of society, not only for taxation. It is used in education, for military service, in health care and by financial institutions, and insurance companies. The role of the Swedish NIN has also made it central to register-based research \citep{scb2016}.

\section{Working with personal identity codes}

The method of validating and extracting information from identification numbers is manually doable and simple in principle but in practice becomes unfeasible with datasets larger than a few dozen observations. The \pkg{hetu} and \pkg{sweidnumbr} packages provide easy-to-use tools for programmatic handling of Finnish and Swedish personal identity codes and Business ID codes.\footnote{In Finnish: Yritys- ja Yhteisötunnus, or Y-tunnus for short, In Swedish: Organisationsnummer} As shown in Table 3, both packages share several core functions and even function names.

\begin{table}[ht]
\centering
\begin{tabular}{lll}
\toprule
    sweidnumbr & hetu & Description \\
  \hline
  rpin & rpin (rhetu) & Generate a vector of random NINs \\
  pin\_age & pin\_age (hetu\_age) & Calculate age from NIN \\
  luhn\_algo & hetu\_control\_char & Calculate check digit / control character from NIN \\
  pin\_ctrl & pin\_ctrl (hetu\_ctrl) & Check NIN validity \\
  pin\_date (pin\_to\_date) & pin\_date (hetu\_date) & Extract Birth date from NIN \\
  pin\_sex & pin\_sex (hetu\_sex) & Extract Sex From NIN \\
  oin\_ctrl & bid\_ctrl & Check OIN/BID validity \\
  roin & rbid & Generate a vector of random OINs/BIDs \\
  
\bottomrule
\end{tabular}
\caption{Exported functions that are shared between both \pkg{sweidnumbr} and \pkg{hetu}. Function alias in parentheses.}
\label{tab:hetu_sweidnumbr_shared_functions}
\end{table}

Both packages utilize R’s efficient vectorized operations, generating and validating over 5 million personal identity codes or Business Identity Codes in less than 10 minutes on a regular laptop. This can meet the practical upper limit set by the current population of Finland (5.5 million people) and Sweden (10.35 million people), providing adequate headroom for handling of relatively large registry datasets.

\subsection{The \pkg{hetu} package}

Printing a data frame containing extracted information in a structured form can be done as follows:

\begin{example}
 > library(hetu)
 > x <- c("010101A0101", "111111-111C", "290201A010M")
 > hetu(x)
\end{example}

The \code{hetu()} function is the workhorse of the \pkg{hetu} package. Without additional parameters it prints out a \code{data.frame} with all information that can be extracted from Finnish NINs as well as a single column that indicates if the NIN is valid as a whole or if it has problems that make it invalid. For demonstration purposes the 3rd NIN we included has an invalid date part; 29th of February would only be a valid date if the year was a leap year, which 2001 is not.

\begin{example}
         hetu    sex p.num ctrl.char       date day month year century valid.pin
1 010101A0101 Female   010         1 2001-01-01   1     1 2001       A      TRUE
2 111111-111C   Male   111         C 1911-11-11  11    11 1911       -      TRUE
3 290201A010M Female   010         M       <NA>  29     2 2001       A     FALSE
\end{example}

The generic way of outputing information found on individual columns is to use the standard \code{hetu()} function with extract-parameter. For example:

\begin{example}
  > hetu("010101A0101", extract = "sex")
  [1] "Female"
  > hetu("010101A0101", extract = "date")
  [1] "2001-01-01"
\end{example}

All column names printed out by the \code{hetu()} function are valid extract parameters. Most commonly used columns have their own function wrappers that are identical in output:

\begin{example}
  > pin_sex("010101A0101")
  [1] "Female"
  > pin_date("010101A0101")
  [1] "2001-01-01"
\end{example}

By importing \CRANpkg{lubridate} \citep{lubridate} functions we have also added \code{pin\_age()} function to calculate the age of individuals in years (default), months, weeks or days. The default option is to calculate age at the current moment but it is also possible to set a specific date as a parameter at which the age is calculated.

\begin{example}
  > pin_age("010101A0101", date = "2004-02-01", timespan = "months")
  The age in months has been calculated at 2004-02-01.
  [1] 37
\end{example}

All NINs passed through the \code{hetu()} function are checked with 10 different tests to determine their validity. The results are crystallized in a single valid.pin column of the \code{hetu()} function output data frame. With the \code{hetu\_diagnostic()} function the user can print the results of these normally hidden tests.

\begin{example}
  > hetu_diagnostic("290201A010M")
  
           hetu is.temp valid.p.num valid.ctrl.char correct.ctrl.char
  1 290201A010M   FALSE        TRUE            TRUE             FALSE
    valid.date valid.day valid.month valid.year valid.length valid.century
  1      FALSE      TRUE        TRUE       TRUE         TRUE          TRUE
\end{example}

When data is inputted manually without validity checks, input errors can creep in. The control character in Finnish personal identity codes combined with validity checks in \pkg{hetu} package can help to catch the most obvious errors. In the example above we can see that the date is incorrect, but also the control character is incorrect. We can simply try three different dates to see if the input error is in the date, month or year part, assuming that the personal number and control character parts were inputted correctly. In this manufactured example the error was in the year part, resulting in the rare leap day date being the correct one.

\begin{example}
  > example_vector <- c("290201A010M", "280201A010M", "290301A010M", "290200A010M")
  > columns <- c("valid.p.num", "valid.ctrl.char", "correct.ctrl.char", "valid.date")
  > hetu_diagnostic(example, extract = columns)
  
           hetu valid.p.num valid.ctrl.char correct.ctrl.char valid.date
  1 290201A010M        TRUE            TRUE             FALSE      FALSE
  2 280201A010M        TRUE            TRUE             FALSE       TRUE
  3 290301A010M        TRUE            TRUE             FALSE       TRUE
  4 290200A010M        TRUE            TRUE              TRUE       TRUE
\end{example}

The \pkg{hetu} package can generate a large number of personal identity codes with the \code{rpin} function. The date range of the generated identity codes can be changed with parameters, but it has a hard coded lower limit at the year 1860 and upper limit in the current date. It has been theorized that the oldest individuals that received a personal identity code in 1960s were born in 1860s. Personal identity codes are never assigned beforehand and therefore it is impossible to have valid personal identity codes that have a future date.

The function can also be used to generate so called temporary personal identity codes. Temporary identity codes are never used as a persistent and unique identifier for a single individual but as a placeholder in institutions such as hospitals when a person does not have a Finnish NIN or the NIN is not known. They can be identified by having a personal number (p.num column in \code{hetu()} function output or NNN as in Table 2) in the range of 900-999.

Here is an example of generating 4 temporary Finnish NINs and checking their validity with \code{pin\_ctrl} function:

\begin{example}
  > set.seed(125)
  > x <- rpin(n = 4, p.male = 0.25, p.temp = 1.0)
  > x
  [1] "250706-9565" "230117-940B" "291247-990E" "130271-908L"
  
  > pin_ctrl(x)
  [1] NA
  
  > pin_ctrl(x, allow.temp = TRUE)
  [1] TRUE TRUE TRUE TRUE
\end{example}

As mentioned earlier, our package also supports similarly generating and checking the validity of Finnish organization identifiers, or Finnish Business ID (BID) numbers. Despite the name, BIDs are used not only for companies and businesses but also for other types of organizations and other juridical persons. Unlike personal identity codes, BIDs do not contain any information about the company. BIDs consist of a random string of 7 numbers followed by a dash and 1 check digit, a number between 0 and 9. 

In addition we have added support for the less known and less widely used numbering scheme for natural persons, Finnish Unique Identification (FINUID) numbers.\footnote{in Finnish: Sähköinen asiointitunniste or SATU for short} FINUID numbers are similar in the sense that they do not contain any biocentric data on the individual and are mainly used by government authorities in IT systems. FINUID numbers consist of 8 numbers and 1 control character calculated in the same way as in personal identity codes. Both of these ID numbers are an example of indexical data, as opposed to Finnish NINs which contain biocentric data in the form of birth date and sex.

\begin{example}
  > bid_ctrl(c("0000000-0", "0000001-9"))
  [1] TRUE TRUE

  > satu_ctrl("10000001N")
  [1] TRUE
\end{example}

The \pkg{hetu} package contains some functions that are not shared with the \pkg{sweidnumbr} package, most notable being \code{hetu()} function. These functions are listed and described in Table 4.

\begin{table}[ht]
\centering
\begin{tabular}{ll}
\toprule
    Function (alias) & Description \\
  \hline
  hetu & Finnish personal identification number extraction \\
  pin\_diagnostic (hetu\_diagnostic) & Diagnostics Tool for HETU \\
  satu\_control\_char & FINUID Number Control Character Calculator \\
  satu\_ctrl & Check FINUID Number validity \\
  
\bottomrule
\end{tabular}
\caption{Functions that are unique to the \pkg{hetu} package and have no equivalent in the \pkg{sweidnumbr} package. Function alias in parentheses.}
\label{tab:hetu_unique_functions}
\end{table}

\subsection{The \pkg{sweidnumbr} package}

The \pkg{sweidnumbr} R package has similar functionality as the \pkg{hetu} package, but for Swedish NINs and with a slightly different syntax. At the time of writing, the package has been downloaded roughly 30 000 times from CRAN\footnote{Source: CRANlogs API, data retrieved at 2022-03-22.}. The example NINs below taken from \citet{sv2007}.

\begin{example}
  > library(sweidnumbr)
  > example_pin <- c("640823-3234", "6408233234", "19640823-3230")
  > example_pin <- as.pin(example_pin)
  > example_pin

  [1] "196408233234" "196408233234" "196408233230"
  Personal identity number(s)
\end{example}

Unlike the \pkg{hetu} package, the \pkg{sweidnumbr} takes advantage of a custom S3 class structure. Therefore the first step is to convert strings with different Swedish NIN formats or numeric variables into a \code{pin} vector using the \code{as.pin()} function. The \code{pin} vector is a S3 object and can be checked is the \code{is.pin()} function.

\begin{example}
  > is.pin(example_pin)

  [1] TRUE
\end{example}

This function only check that the vector is a \code{pin} object, but not if the actual NIN are valid. To check the Swedish NIN using the control numbers, or check digits, we simply use the \code{pin\_ctrl()} function.

\begin{example}
  > pin_ctrl(example_pin)

  [1]  TRUE  TRUE FALSE
\end{example}

Just as in the \pkg{hetu} package we can extract information from the Swedish NIN with specialized functions. We can now use \code{pin\_birthplace()}, \code{pin\_sex()}, and \code{pin\_age()} to extract information on birthplace, sex, and age.

\begin{example}
  > pin_sex(example_pin)
   
  [1] Male Male Male
  Levels: Male

  > pin_birthplace(example_pin)
   
  [1] Gotlands län Gotlands län Gotlands län
  28 Levels: Stockholm stad Stockholms län Uppsala län ... Born after 31 december 1989

  > pin_age(example_pin)
  [1] 55 55 55
  > pin_age(example_pin, date = "2000-01-01")
  [1] 35 35 35
\end{example}

As with the \pkg{hetu} R package we can also generate, or simulate, NINs with the \code{rpin()} function. Shared functions exist also for Swedish organization identifiers, or Swedish organizational numbers (SON), in the form of \code{as.oin()}, \code{is.oin()}, and \code{oin\_ctrl()} functions. Unlike the Finnish BID, the \code{oin} contain information on the type of organization of a given SON, which can be determined by using the \code{oin\_group()} function.

\begin{example}
  > oin_group(example\_oin)

  [1] Aktiebolag
  [2] Stat, landsting, kommuner, församlingar
  [3] Ideella föreningar och stiftelser
  3 Levels: Aktiebolag ... Stat, landsting, kommuner, församlingar
\end{example}

Similar to \pkg{hetu} package \code{rbid()} function and \pkg{sweidnumbr} \code{rpin()} function for natural persons, we can generate new SONs using the \code{roin()} function.

\begin{example}
  > set.seed(125)
  > roin(3)

  [1] "776264-6144" "274657-0148" "827230-7631"
  Organizational identity number(s)
\end{example}

Due to national characteristics of Swedish numbering schemes for natural and juridical persons there are some functions that are unique to the \pkg{sweidnumbr} packages. These functions are listed in Table 5.

\begin{table}[ht]
\centering
\begin{tabular}{ll}
\toprule
    Function & Description \\
  \hline
  as.oin & Parse organizational identity numbers \\
  as.pin & Parse personal identity numbers to ABS format \\
  fake\_pins & Fake personal identity numbers and names \\
  format\_pin & Formatting pin \\
  is.oin & Test if a character vector contains correct 'oin' \\
  is.pin & Parse personal identity numbers to ABS format \\
  oin\_group & Calculate organization group from 'oin' \\
  pin\_birthplace & Calculate the birthplace of 'pin' \\
  pin\_coordn & Check if 'pin' is a coordination number \\
\bottomrule
\end{tabular}
\caption{Functions that are unique to the \pkg{sweidnumbr} package and have no equivalent in the \pkg{hetu} package.}
\label{tab:hetudiagnostics}
\end{table}

\section{Discussion}

The \pkg{hetu} and \pkg{sweidnumbr} R packages provides a free and open source methods for validating and extracting data from a large number of Finnish and Swedish national identity numbers (NIN). While the package's target audience consists mostly of Finnish and Swedish users and people with particular interest in NIN systems around the world, the package makes a generic contribution in developing methodologies related to NIN handling in R, and more generally for {\it structured data} in the field of computational humanities (see e.g. \citep{makela2020}). In the future, a more generic package or class structures could be useful to unify the handling of different NIN systems around the world, and might be worth exploring further.\footnote{Although in Unix philosophy one stated aim is "Write programs that do one thing and do it well" and "Write programs to work together" as opposed to writing one program that aims to do everything. The value in one generic program might mostly be related to sharing information and good coding practices among interested parties around the world.}

The origins of these packages can be traced to early 2010s when one curious individual wanted to analyze a large number of Finnish NINs that were leaked to the internet by an anonymous hacker. The legality and morality of handling such dataset containing personal information was and is in a grey area at best. As developers of this package we cannot condone such activities, even if they are conducted out of curiosity and not ill intentions, but we acknowledge that we cannot prevent our users from doing that either.\footnote{The use of "Good, not evil" license texts is thought to be problematic and against the ethos of open source software} 

We have acknowledged beforehand that random NINs generated with the \pkg{hetu} and \pkg{sweidnumbr} packages could theoretically be used for purposes such as synthetic identity fraud.\footnote{see \citet[32]{brensinger2021} for a short description of synthetic fraud related to American SSNs} On the other hand it is important to note that such NINs could also be created by hand as the same information that we have used in developing our algorithms is available at Finnish authorities' web page \citep{hetudvv}. Our package can be useful for many, and it does not make fraudulent activities significantly easier for malevolent individuals which is essential in judging the pros and cons of releasing this software to the public.

Similar data breaches have made people more wary of digital services. Privacy concerns can push Finland, Sweden and other Nordic countries towards redesigning their national identification numbers to omit the embedded personal information sometime in the future. There is an ongoing government project led by Ministry of Finance to redesign the system of Finnish NINs \citep{hetuvm}. These and other related policy and legislation changes will be closely monitored and, if necessary, the package functions can be adjusted accordingly.

Both packages are published under permissive BSD 2-clause license. We encourage users to study the source code and modify it for their own use or submit improvements to our public code repositories.\footnote{\url{https://github.com/rOpenGov/hetu}, \url{https://github.com/rOpenGov/sweidnumbr}}

\section{Acknowledgements}

We are grateful to all contributors, in particular Juuso Parkkinen and Joona Lehtomäki for their support in the initial package development. This work is part of rOpenGov\footnote{\url{http://ropengov.org}} and contributes to the FIN-CLARIAH research infrastructure for computational humanities. LL and PK were supported by Academy of Finland (decisions 295741, 345630).

\bibliography{hetu_bib}

\address{Pyry Kantanen\\
  Department of Computing\\
  PO Box 20014 University of Turku\\
  Finland\\
  ORCiD: 0000-0003-2853-2765\\
  \email{pyry.kantanen@utu.fi}}
  
\address{Måns Magnusson\\
  Department of Statistics\\
  Uppsala University\\
  Sweden\\
  ORCiD: 0000-0002-0296-2719\\
  \email{mans.magnusson@statistik.uu.se}}

\address{Leo Lahti\\
  Department of Computing\\
  PO Box 20014 University of Turku\\
  Finland\\
  ORCiD: 0000-0001-5537-637X\\
  \email{leo.lahti@utu.fi}}